\documentclass[11pt]{article}

\usepackage{booktabs}
\usepackage{dcolumn} 
\usepackage{epstopdf}
\usepackage{fourier}
\usepackage{fullpage}
\usepackage{graphicx}
\usepackage{hyperref}
\usepackage{longtable} 
\usepackage{natbib}
\usepackage{rotating}
\usepackage{tabularx}
\usepackage{amsmath}
\usepackage{algorithmic} 
\usepackage{algorithm2e}

\hypersetup{
  colorlinks = TRUE,
  citecolor=blue,
  linkcolor=red,
  urlcolor=black
}

\newcommand{\important}[1]{\textcolor{blue}{\textbf{ #1}}}
\newcommand{\quantclaim}[1]{\textcolor{red}{\textbf{ #1}}}
\newcommand{\starlanguage}{Some TK scale about statistical significance.}

\begin{document} 

\title{ {{project_name}} }

\date{\today}

\author{ {{ author }} \\ {{ school }} \footnote{ {{ footnote }} } }
\maketitle

\begin{abstract}
\noindent  Here is a really great abstract.  \newline
\noindent JEL J01, J24, J3
\end{abstract} 

\section{Introduction}
\cite{smith1999wealth} had some great ideas! 

\important{This is an important claim!}
\quantclaim{This is a quantitative claim!} 
There a TK important claims in this document. 

\subsection{Plots!}

\begin{figure}[h]
\centering 
\caption{Here is a figure} \label{fig:hist}
\begin{minipage}{0.50 \linewidth}
\includegraphics[width = \linewidth]{./plots/hist.pdf}
\\
\emph{Notes}: Here are some notes.
\end{minipage} 
\end{figure}



\bibliographystyle{aer}
{{ bibliography_line }}

\end{document} 

