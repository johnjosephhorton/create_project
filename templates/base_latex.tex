\documentclass[11pt]{article}

\usepackage{booktabs}
\usepackage{dcolumn} 
\usepackage{epstopdf}
\usepackage{fourier}
\usepackage{fullpage}
\usepackage{graphicx}
\usepackage{hyperref}
\usepackage{longtable} 
\usepackage{natbib}
\usepackage{rotating}
\usepackage{tabularx}
\usepackage{amsmath}
\usepackage{algorithmic} 
\usepackage{algorithm2e}

\hypersetup{
  colorlinks = TRUE,
  citecolor=blue,
  linkcolor=red,
  urlcolor=black
}

\begin{document} 

\title{ {{project_name}} }

\date{\today}

\author{ {{ author }} \\ {{ school }} \footnote{ {{ footnote }} } }
\maketitle

\begin{abstract}
\noindent  Here is a really great abstract.  \newline
\noindent JEL J01, J24, J3
\end{abstract} 

\section{Introduction}
\cite{smith1999wealth} had some great ideas! 

\section{Getting some stuff done in R}
According to R's calculations, $1 + 1$ is equal to:

%\input{./numbers/tough_problem.txt}

\subsection{Plots!}

% \begin{figure}[h]
%   \centering
%   \includegraphics[scale=0.25]{./plots/hist.png}
%   \caption{Here is a figure}
%   \label{fig:hist}
% \end{figure}

\subsection{We can make R get data from our database}

%\input{./numbers/sql_output.txt}

\section{Using matplotlib for making figures}

% \begin{figure}[h]
%   \centering
%   \includegraphics[scale=0.25]{./diagrams/matplotlib.png}
%   \caption{Here is a matplot lib constructed figure}
%   \label{fig:matplotlib}
% \end{figure}


\bibliographystyle{aer}
{{ bibliography_line }}

\end{document} 

